\chapter{Menggunakan Git dengan GitHub}

\section{Membuat Repository Private/Public}
GitHub memungkinkan pembuatan repository baik private maupun public untuk mengelola proyek. Berikut adalah langkah-langkah untuk membuat repository di GitHub:
\begin{enumerate}
	\item Masuk ke akun GitHub dan klik tombol \texttt{New Repository} di halaman utama atau navigasi profil.
	\item Berikan nama repository dan deskripsi opsional.
	\item Pilih apakah repository tersebut akan bersifat \texttt{Public} (dapat diakses oleh siapa saja) atau \texttt{Private} (hanya bisa diakses oleh pemilik dan kolaborator yang diizinkan).
	\item Pilih apakah ingin menambahkan file \texttt{README.md}, \texttt{.gitignore}, atau lisensi.
	\item Klik \texttt{Create repository}.
\end{enumerate}

Setelah repository dibuat, URL Git untuk repository tersebut akan diperoleh, yang dapat digunakan sebagai remote di proyek lokal.

\section{Menambahkan Remote dan Push Perubahan}
Setelah repository di GitHub dibuat, langkah selanjutnya adalah menambahkan repository tersebut sebagai \textit{remote} di proyek Git lokal. Berikut langkah-langkahnya:

\subsection{Menambahkan Remote}
Untuk menambahkan remote ke repository lokal, gunakan perintah \texttt{git remote add} dengan URL GitHub dari repository.
\begin{lstlisting}[language=bash]
	git remote add origin https://github.com/username/nama-repo.git
\end{lstlisting}
Pada perintah ini, \texttt{origin} adalah nama default untuk remote repository utama. Nama lain dapat digunakan jika diinginkan.

\subsection{Push Perubahan ke Remote}
Setelah remote ditambahkan, perubahan dari repository lokal dapat dikirim ke repository di GitHub menggunakan perintah \texttt{git push}.
\begin{lstlisting}[language=bash]
	git push -u origin main
\end{lstlisting}
Perintah ini akan mengirim semua commit di cabang \texttt{main} ke remote \texttt{origin} (GitHub). Opsi \texttt{-u} digunakan untuk mengatur \texttt{main} sebagai cabang default untuk operasi \texttt{push} berikutnya.

\subsection{Sinkronisasi Perubahan}
Untuk menarik (pull) perubahan dari repository remote ke lokal, gunakan perintah:
\begin{lstlisting}[language=bash]
	git pull origin main
\end{lstlisting}
Perintah ini memastikan repository lokal tetap sinkron dengan repository GitHub.
