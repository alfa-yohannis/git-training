\chapter{Penggunaan Git Dasar}

\section{Git Init: Menginisialisasi Repository}
Perintah \texttt{git init} digunakan untuk menginisialisasi repository Git baru. Perintah ini akan membuat direktori \texttt{.git} di dalam folder proyek, yang akan menyimpan semua metadata dan riwayat version control.
\begin{lstlisting}[language=bash]
	git init
\end{lstlisting}
Setelah perintah ini dijalankan, folder proyek sekarang menjadi repository Git dan siap untuk melacak perubahan.

\section{Git Add: Menambahkan Perubahan ke Staging}
Perintah \texttt{git add} digunakan untuk menambahkan perubahan pada file ke area staging, yaitu tahap sebelum perubahan tersebut dikomit ke repository. File individu atau semua file yang telah diubah bisa ditambahkan sekaligus.
\begin{lstlisting}[language=bash]
	git add <nama_file>
	git add .
\end{lstlisting}
Menggunakan \texttt{git add .} akan menambahkan semua file yang berubah ke staging.

\section{Git Commit: Mengkomit Perubahan}
Setelah perubahan ditambahkan ke area staging, perintah \texttt{git commit} digunakan untuk mengunci perubahan ke dalam repository. Setiap \texttt{commit} memerlukan pesan yang menjelaskan perubahan yang dilakukan.
\begin{lstlisting}[language=bash]
	git commit -m "Deskripsi perubahan"
\end{lstlisting}
Pesan commit harus jelas dan menjelaskan apa yang telah diubah untuk memudahkan pelacakan riwayat perubahan.

\section{Git Reset: Menghapus Perubahan dari Staging}
Jika ingin membatalkan perubahan yang telah ditambahkan ke staging tanpa mengubah file yang sebenarnya, gunakan perintah \texttt{git reset}.
\begin{lstlisting}[language=bash]
	git reset <nama_file>
\end{lstlisting}
Perintah ini akan menghapus file dari staging, namun perubahan pada file tetap ada. Untuk membatalkan perubahan pada file itu sendiri, gunakan \texttt{git checkout}.

\section{Penggunaan File .gitignore}
File \texttt{.gitignore} digunakan untuk menentukan file atau direktori mana yang tidak ingin dilacak di dalam repository Git. Pola atau nama file yang tidak akan ditambahkan ke staging meskipun diubah bisa ditentukan.
\begin{lstlisting}[language=bash]
	# Contoh file .gitignore
	node_modules/
	*.log
	*.tmp
\end{lstlisting}
Dengan menambahkan pola file ke \texttt{.gitignore}, Git akan mengabaikan file-file tersebut.

\section{Membuat dan Menggabungkan Branch}
Git memungkinkan pekerjaan pada berbagai fitur atau perbaikan secara terpisah menggunakan \textit{branch}. Perintah \texttt{git branch} digunakan untuk membuat cabang baru, dan \texttt{git checkout} untuk berpindah ke cabang tersebut.
\begin{lstlisting}[language=bash]
	git branch <nama_branch>
	git checkout <nama_branch>
\end{lstlisting}
Setelah perubahan selesai, cabang tersebut bisa digabungkan ke dalam cabang utama (\texttt{main} atau \texttt{master}) menggunakan perintah \texttt{git merge}.
\begin{lstlisting}[language=bash]
	git checkout main
	git merge <nama_branch>
\end{lstlisting}
Proses ini akan menggabungkan perubahan yang dilakukan pada cabang lain ke dalam cabang utama.
